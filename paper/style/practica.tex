%----------------------------------------------------------------------------------------
%	PAQUETES Y OTRAS CONFIGURACIONES
%----------------------------------------------------------------------------------------

\documentclass[paper=letter, fontsize=11pt]{scrartcl} % Tamaño de papel y letra para el documento

\usepackage[utf8]{inputenc} % Los caracteres acentuados se pueden escribir normalmente en el código
\usepackage[T1]{fontenc} % Configuración de fuente de salida
\usepackage{fourier} % Se usa una fuente diferente al default
\usepackage[spanish,es-noquoting]{babel} % Se configura como documento en español
\usepackage{amsmath,amsfonts,amsthm} % Paquetes para escribir formulas matemáticas
\usepackage{graphicx} % Paquetes para incluir imágenes
\usepackage{hyperref}

\usepackage[fixlanguage]{babelbib}
\selectbiblanguage{spanish}

\usepackage{circuitikz}
\usepackage{tikz}
\usetikzlibrary{arrows}

\usepackage{sectsty} % Paquete para configuración de secciones
\allsectionsfont{\centering \normalfont \scshape} % Los títulos de las secciones son centrados, con la misma fuente y pequeñas mayúsculas

\usepackage{todonotes}
\usepackage{microtype}

\usepackage{fancyhdr} % Paquete para personalizar pies y cabeceras de página
\pagestyle{fancyplain} % Todas las páginas con las mismas cabeceras y pies de página
\fancyhead{} % Sin cabecera
\fancyfoot[L]{} % Vacío en la izquierda del pie de página
\fancyfoot[C]{} % Vacío en el centro del pie de página
\fancyfoot[R]{\thepage} % Número de página en el pie de pagina
\renewcommand{\headrulewidth}{0pt} % Sin lineas en la cabecera
\renewcommand{\footrulewidth}{0pt} % Sin lineas en el pie de página
\setlength{\headheight}{13.6pt} % Altura de cabecera

\numberwithin{equation}{section} % Numera ecuaciones en cada sección
\numberwithin{figure}{section} % Numera figuras en cada sección
\numberwithin{table}{section} % Numera tablas en cada sección

\setlength\parindent{0pt} % Quita la indentación de los párrafos

\newcommand{\horrule}[1]{\rule{\linewidth}{#1}} % Comando personalizado para hacer linea horizontal
