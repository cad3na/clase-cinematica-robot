%-------------------------------------------------------------------------------
%	PAQUETES Y OTRAS CONFIGURACIONES
%-------------------------------------------------------------------------------

\input{../style/practica.tex}

%-------------------------------------------------------------------------------
%	TITULO
%-------------------------------------------------------------------------------

\title{
	\normalfont \normalsize
	\begin{figure}[h]
		\begin{center}
			\includegraphics[width=0.3\textwidth]{../images/UNITEC.png}
		\end{center}
	\end{figure}
	\textsc{Cinemática del Robot} \\ [25pt]
	\horrule{0.5pt} \\[0.4cm] % Linea horizontal delgada
	\huge Reporte de prácticas para materia presencial. \\ % Titulo de la práctica
	\horrule{2pt} \\[0.5cm] % Linea horizontal mas gruesa
}

\author{Roberto Cadena Vega} % Nombre del profesor

\date{\normalsize \today} % Fecha de la práctica

%-------------------------------------------------------------------------------
%	EMPIEZA EL DOCUMENTO
%-------------------------------------------------------------------------------

\begin{document}

\maketitle % Imprime el título

%-------------------------------------------------------------------------------
%	PRACTICAS
%-------------------------------------------------------------------------------

\section{Prácticas}

	El indice básico de las prácticas es:

	\begin{enumerate}
		\item Introducción a Jupyter
		\item Transformaciones homogeneas
		\item Visualización de sistemas mecánicos
		\item Algoritmo de Denavit - Hartenberg
		\item Cinematica directa
		\item Cinemática inversa
	\end{enumerate}

	Estas prácticas se pueden encontrar en el repositorio principal en linea de la matería\cite{github:cinematica}.

%-------------------------------------------------------------------------------
%	INVESTIGACION PREVIA
%-------------------------------------------------------------------------------

\section{Investigaciónes previas}

	Los temas a investigar por parte del alumno previo a cada práctica son:

	\begin{enumerate}
		\item Propiedades de matrices de transformación\cite{apuntesdin:2014}.
		\item Transformaciones homógeneas\cite{apuntescin:2014}.
		\item Cinemática directa de pendulo doble\cite{apuntesdin:2014}.
		\item Denavit-Hartenberg\cite{robotica:2005}.
		\item Manipulador PUMA\cite{apuntescin:2014}.
		\item Cinemática inversa de pendulo doble\cite{apuntescin:2014}.
	\end{enumerate}

%-------------------------------------------------------------------------------
%	OBJETIVOS
%-------------------------------------------------------------------------------

\section{Objetivos}

	El objetivo general de las prácticas es familiarizar al alumno con las librerías de computo cientifico necesarias para la simulación de sistemas mecánicos como los robots manipuladores, utilizando una opción de código libre y acceso libre (sin costo y sin restricciones comerciales), que a la vez es ampliamente utilizada en academia y en industria.

	Los objetivos por práctica son:

	\begin{enumerate}
		\item El alumno implementará código computacional para obtener la solución a ecuaciones matriciales.
		\item El alumno implementará código computacional para obtener la posición de manipuladores robóticos.
		\item El alumno implementará código para graficar y animar el comportamiento de un sistema mecánico.
		\item El alumno implementará código para calcular posiciones de cuerpos rigidos despues de aplicarseles transformaciones homogéneas.
		\item El alumno implementará código para animar el comportamiento de un robot manipulador.
		\item El alumno implementará código para obtener los angulos de un manipulador a partir de una interfaz gráfica.
	\end{enumerate}

%-------------------------------------------------------------------------------
%	TIEMPO
%-------------------------------------------------------------------------------

\section{Tiempo de realización}

	El tiempo de realización de cada práctica es de $2$ sesiones de $2$ horas cada una.

	Tomando en cuenta una sesion de $2$ horas a la semana de prácticas en laboratorio, los alumnos deben ser capaces de terminar las prácticas de laboratorio en la semana 13 y entregar en la semana 14 de clases.

%-------------------------------------------------------------------------------
%	MARCO TEORICO
%-------------------------------------------------------------------------------

\section{Marco teórico}

	En la actualidad existen una cantidad importante de cursos en linea\cite{google:Robotics} y presenciales que utilizan la programación como medio de reforzamiento a la teoría matemática de los robots manipuladores, sin embargo la gran mayoría existe en otro idioma (principalmente ingles) y el lenguaje de programación predominante es MATLAB y Simulink\cite{MATLAB:2015}, por lo que existe una buena motivación para crear prácticas ad-hoc para este curso.

	Cabe notar que el nivel de conocimientos previos de la mayoria de los cursos que se encuentran en linea es mas elevado del que se establece para esta materia, por lo que tambien es importante que se consideren las limitaciones de los alumnos, especialmente porque esta materia es de quinto cuatrimestre.

	De la misma manera se ofrece una explicación en linea\cite{github:instalacion} para la instalación del software necesario para la implementación del código de las prácticas, en un intento de nivelar desigualdades de conocimientos informaticos necesarios para el computo cientifico-tecnológico.

%-------------------------------------------------------------------------------
%	RESULTADOS
%-------------------------------------------------------------------------------

\section{Resultados}

	En ocasiones anteriores se han encontrado tanto errores en la programación inicial realizada por el profesor, tanto como ambiguedades en el lenguaje utilizado, lo que da pie a errores comúnes en la implementación por parte del alumno, por lo que es importante mantener una filosofía de mejora permanente en la implementación inicial realizada por el profesor, lo cual trae como ventajas añadidas el tomar en cuenta nuevos paradigmas de programación que pudieran resultar mas intuitivos para los alumnos, asi como implementaciones mas atrayentes para los alumnos.

%-------------------------------------------------------------------------------
%	BIBLIOGRAFÍA
%-------------------------------------------------------------------------------

{\bibliographystyle{plain}}

\bibliography{bibliografia}

%-------------------------------------------------------------------------------
%	FIN DEL DOCUMENTO
%-------------------------------------------------------------------------------

\end{document}
